\capitulo{7}{Lineas de trabajo futuras}



Como líneas futuras de este proyecto se proponen las siguientes:
\begin{enumerate}
    \item [-] En primer lugar, el empleo de este estudio como base para aplicar modelados matemáticos más complejos u otros no mencionados como el de Jonh Sorensen, modelos que incluyan el retraso, o el Simulador Metabólico de Diabetes tipo 1(T1DMS), o bien modelos novedosos por desarrollar. Estos modelos podrían proporcionar una comprensión más completa de la fisiopatología de la diabetes y permitir la simulación de diferentes escenarios clínicos y tratamientos.
    \item[-] Continuar con el estudio de los mecanismos de control de la glucosa a partir del algorimto de control clásico PID propuesto, explorando enfoques más sofisticados y precisos. Esto podría llevarse a cabo bien analizando reguladores ya creados o bien desarrollando unos nuevos, adaptándose mejor a la naturaleza dinámica y no lineal (como se ha considerado en alguna ocasión para este estudio) del sistema glucorregulatorio. En este apartado se encontraría incluso el desarrollo de glucómetros que permita la lectura de valores de glucosa muy elevados (pues en muchos de ellos, cuando el sensor detecta un límite máximo determinado,deja de captar niveles de concentración superiores). Entre los sistemas que propongo se encuentran los reguladores de control predictivo (que preveen el comportamiento futuro de la glucosa y calcula la señal de control a partir de funciones de costo predefinidas), los de control adaptativo (que ajustan los parámetros en función de las condiciones del sistema de forma continua), o el regulador feedfordward, que se anticipa y compensa las perturbaciones antes de que afecten al sistema (a los niveles de glucosa, en este caso). A diferencia de los anteriores, este regulador no actúa en respuesta a una señal de error, sino que ajusta la señal de manera proactiva. 
    \item [-] Estudiar de manera más detallada y realista el modelado de los diferentes tipos de insulina exógena, considerando su cinética de absorción y acción, analizando cómo se liberan en el torrente sanguíneo y cómo es posible modelar estas relaciones. Esto mejoraría la personalización del tratamiento insulinoterapéutico, así como contribuiría a una optimización de las estrategias de dosificación de la insulina.
    \item[-] Abordar el efecto de otras variables de interés en la interacción de la glucosa e insulina, como el estrés, la calidad del sueño, o la ansiedad, y tratar de obtener relaciones entre ellas. Respecto a la más interesante para mí, el estrés, se conoce que la liberación del cortisol y adrenalina aumenta los niveles de glucosa en sangre al promover la liberación de la glucosa almacenada en el hígado y al reducir la sensibilidad a la insulina. El estrés crónico aumenta el riesgo a desarrollar Diabetes Tipo 2, y puede incluso causar el agotamiento de las células beta del páncreas.
    \item[-] Combinar el modelado matemático llevado a cabo con nuevas tecnologías y enfoques, para mejorar el monitoreo de la diabetes, así como unir este campo con la inteligencia artificial. Este análisis podría servir de base para la introducción a la ingeniería de tejidos pancreáticos o las terapias con células beta.
\end{enumerate}