\capitulo{1}{Introducción}


La interacción entre la glucosa y la insulina en el organismo es un mecanismo biológico complejo cuyo correcto funcionamiento es fundamental para el bienestar y la vida de la sociedad. Alteraciones en esta interacción pueden desencadenar grandes riesgos, como es el caso de las patologías cardiovasculares o renales. Especial relevancia cobra en este estudio la enfermedad de la Diabetes Mellitus, caracterizada por la ausencia o falta de insulina en el cuerpo, que desencadena altos niveles glucémicos que suponen significativos riesgos para la salud.

En este proyecto se llevará a cabo un análisis del sistema glucorregulatorio, adecuado y alterado, a partir de modelos matemáticos que han ido surgiendo en las últimas décadas, que tratan de reflejar el comportamiento de las variables de la forma más próxima posible a la realidad. Además, en base a estos modelos es posible simular dinámicas de la glucosa ante entradas al sistema, como la administración de insulina exógena, así como estudiar el efecto de algunas perturbaciones como la ingesta o el ejercicio físico.

El avance en el conocimiento sobre estos mecanismos es clave para el desarrollo de dispositivos médicos que puedan contrarrestar los efectos nocivos en caso de haberlos, o bien para la mejora de la calidad de vida de los pacientes. Los sistemas de monitorización de glucosa, o sistemas más sofisticados, como el páncreas artificial, suponen mejoras revolucionarias e innovadoras que se traducen en grandes resultados para el tratamiento de la Diabetes Mellitus, por lo que también se ha considerado la inclusión de sencillos sistemas de control en el proyecto.

La gran parte de las simulaciones y experimentos llevados a cabo se encuentran en el apartado de resultados, mientras que las secciones Conceptos Teóricos y Metodología reúnen la información necesaria para obtenerlas. Es resto de la información relevante, para finalizar, está ubicada en el anexo.




