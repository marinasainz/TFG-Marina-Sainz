\capitulo{6}{Conclusiones}

Se recoge en este proyecto un análisis del comportamiento de la glucosa para diferentes situaciones, así como el efecto que producen en ella. 

Se ha estimado que la diabetes presenta cada vez más opciones terapéuticas de control para alcanzar la normoglucemia, y que muchas de ellas pasan por el análisis matemático de variables, cuya estimación es clave para el correcto funciomiento del sistema regulatorio aplicado. Los parámetros incluidos en estos modelos presentan valores muy específicos que han sido establecidos con el máximo rigor para evitar la aparición de riesgos en los pacientes. Se ha mostrado cómo leves variaciones en ellos pueden desembocar en fallos catastróficos en el sistema de control. Además, se ha querido mostrar la complejidad a la que tienen que hacer frente estos sistemas, que están condicionados a su vez por diversas variables que son únicas y diferentes para cada paciente. Se ha pretendido estudiar también el efecto de la combinación de diferentes opciones, entre las que ha destacado la efectividad del deporte, especialmente después de las comidas, logrando una hipotética disminución hasta en 2/3 de la dosis de insulina administrada, pese a que su marco debe ser regulado para cada paciente debido al riesgo de hipoglucemias. La combinación de diferentes acciones de insulina se ha llevado a cabo como simulaciones experimentales, con el fin de comprobar si sería posible reducir ciertas dosis de insulina prolongada si se combina con insulina rápida, cuyos resultados han sido favorables. Estos resultados son solo hipótesis, y se pretende reivindicar desde esta sección la importancia de la personalización de los tratamientos. Sin embargo, la terapia insulínica, que ha sido presentada de unos años a la actualidad como la solución más efectiva y eficaz a esta patología, comienza a verse desplazada y sustituida por grandes dispositivos de control que manejan de forma automática las descompensaciones de glucosa y administran la insulina necesaria. El efecto que esta acción produce se ha tratado de reflejar en los reguladores sintonizados, que, aunque de manera muy simple, se ha logrado la estabilización de la glucosa en su valor basal, así como la disminución de la amplitud de la curva y del tiempo de asentamiento. 
Así, la creación de grandes dispositivos de control, como el páncreas artificial, regidos por sistemas regulatorios y estimaciones matemáticas, auguran mejoras significativas para el tratamiento de la diabetes, así como en la calidad de vida para los pacientes.


\section{Aspectos relevantes}

Respecto a los aspectos más relevantes del proyecto, se pueden resumir en los siguientes puntos:
\begin{enumerate}
    \item El interés de este proyecto radica, a mi parecer, en la combinación de estrategias de ingeniería con conocimiento biológico para el desarrollo de nuevas estrategias que permitan avanzar en la diabetes. La finalidad de este proyecto no ha sido observar la variación de la glucosa a nivel biológico, sino de entender que su comportamiento puede ser regido por relaciones matemáticas que tratan de aproximarse lo máximo posible a la realidad, y que a su vez pueden ser empleadas para la obtención de soluciones médicas.
    Se trata este proyecto como el inicio del estudio del sistema glucorregulatorio para futuros estudiantes que partan de estos resultados para obtener otros nuevos. Especial relevancia adquiere para mí la modelización que se ha llevado a cabo en otros artículos mencionados a lo largo de esta memoria de funciones como la ingesta, o el ejercicio físico, cuyo comportamiento se ha estimado de manera matemática para permitir observar su efecto en la glucosa. 
    Sin embargo, a nivel insulínico, he encontrado este tema complejo, pues mediante las ecuaciones diferenciales aplicadas que simplifican profundamente el comportamiento del sistema, no se logra modelar de forma óptima algunos casos, o bien el comportamiento de la glucosa no se ajusta del todo a la realidad.
    Especial sorpresa ha tenido para mi encontrarme con una simple ecuación que represente el comportamiento del páncreas, aunque sea de forma aproximada, dándome a entender que este campo es verdaderamente complejo y busca relaciones matemáticas muy precisas.
    \item Inicialmente se estudiaron los modelos matemáticos clásicos de la glucosa, y tras una búsqueda exhaustiva, se determinó trabajar con los Modelos De Bergman y Ackerman. Tras unas pruebas iniciales mediante la inclusión de variables y pruebas de datos diferentes, se determinó que la mejor opción era el Modelo de Bergman, pues presentaba más información y ampliaciones. El Modelo de Ackerman presentó al inicio resultados confusos para variables como el ejercicio, por lo que se decidió desestimarlo para estos casos. De esta manera, se decidió emplear únicamente el Modelo de Bergman, pues para casos iniciales, sería redundante presentar el mismo comportamiento de la glucosa siguiendo dos modelos diferentes. El acceso a los modelos matemáticos de la glucosa ha resultado ser bastante restringido, lo que en ocasiones ha limitado el estudio.
    \item Una vez seleccionado el Modelo, se comenzó con el análisis de parámetros y de su comportamiento. Pese a la definición de las ecuaciones por Bergman en \cite{bergman1979quantitative}, se han encontrado complicaciones a nivel matemático, pues, desde entonces, el Modelo ha avanzado mucho, como en el propuesto en \cite{fisher1991semiclosed}, donde no se termina de aclarar el procedimiento matemático seguido. Además, el valor de los parámetros del modelo se ha establecido que fueran los seleccionados en el artículo anteriormente mencionado, lo que invita a pensar que tienen un carácter general y que no son personalizados para cada paciente, de ahí el análisis realizado en este estudio sobre ellos.
    \item En cuanto a la administración de insulina exógena, se ha realizado un apartado estudiando la combinación de diferentes estrategias para pacientes diabéticos. Pese a que se ha tratado solo de un marco hipotético, me ha resultado sorprendente visualizar el efecto de los tipos de insulinas según su forma de acción, mostrando cómo la administración de insulina constante (a la que se ha considerado insulina prolongada), causa una reducción general de los niveles de glucosa en el organismo; mientras que la insulina de acción rápida ejerce un efecto más significativo pero en menores intervalos de tiempo. Sin embargo, este apartado se ha visto condicionado por la poca distinción entre tipos de insulina externa que he encontrado en refeencia a modelos matemáticos. Cabe remarcar que para este apartado se ha reflejado \textit{el efecto} de estas combinaciones en la glucosa, pero su modelado no se corresponde con la realidad. Para el caso de la insulina prolongada, se ha determinado una insulina constante para todo el día, comportamiento que no es del todo cierto, puesto que, pese a que este tipo de insulina presenta una duración de acción de 18 - 24 h, no supone una administración constante de insulina. Por otro lado, la insulina de acción rápida se ha modelado como una rampa, donde se ha calculado la pendiente de la curva de la insulina para lograr en el tiempo de pico de acción el nivel máximo, mientras que luego esos niveles se van reduciendo hasta desaparecer (siguiendo la duración de esta insulina en el organismo). Este comportamiento en la realidad se correspondería con un paciente al que se le está administrando la insulina de manera continua y progresiva en vez de con una dosis puntual. Sin embargo, el efecto en la glucosa si que se ha considerado el deseado.
    \item Respecto a la combinación de estrategias, se ha realizado también de manera hipotética un experimento con el fin de comparar la administración de una dosis mayor de insulina prolongada y la administración de esta insulina en menor dosis, combinada con el ejercicio y la insulina rápida. La finalidad de esta simulación ha sido determinar que existen factores que dependen del paciente, como el deporte, que sí que pueden contribuir a reducir los niveles de glucosa, disminuyendo la necesidad de alternativas de administración de insulina. Lejos de desprestigiar las dosis insulínicas, se pretende con este apartado mostrar las alternativas y el efecto de diferentes combinaciones en el organismo.
    \item Mediante los sistemas de control estudiados y sintonizados se ha comprobado su eficacia, y la extensión de esta eficacia a grandes dispositivos que facilitan la calidad de vida de muchos pacientes. Llamativa ha sido la comparación de los reguladores obtenidos con el mecanismo de una persona sana, dando a entender que el comportamiento de la glucosa varía para cada individuo y que se pueden conseguir tan buenos resultados a nivel glucorregulatorio como los que se obtienen con un sistema no alterado.
    \item Por último, respecto al testeado del modelo con datos reales, hubo una puesta en contacto personal con el Servicio de Nutrición del Hospital de Burgos (HUBU) para la solicitud de informes diabéticos, pero tras redactar el Informe de la Comisión de Bioética, no se logró realizar un acuerdo.
\end{enumerate}


