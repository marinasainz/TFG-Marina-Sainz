\capitulo{2}{Objetivos}

Este trabajo surge de la inquietud en cuanto a la comprensión de los mecanismos biológicos y el desarrollo de estrategias y planes de control que aparecen para resolver los fallos de estos mecanismos. La complejidad de estas relaciones hace que sea necesario un alto conocimiento y precisión sobre ellas, especialmente en estos casos, en los que está involucrado el funcionamiento del propio cuerpo humano, y donde los riesgos pueden ser altamente perjudiciales, como es el caso de la Diabetes Mellitus.

\section{Objetivos del proyecto}

\begin{enumerate}
    \item Estudio del comportamiento del organismo, analizando la interacción de la glucosa e insulina en base al modelado matemático. Estos modelos cuentan con parámetros que reflejan interacciones de la glucosa, por lo que el análisis de su variación también se consolida como subobjetivo de este apartado.
    \item Estudio del comportamiento alterado del organismo, reflejado en la diabetes mellitus, y sus consecuencias en el sistema glucorregulatorio. Mediante un modelo modificado, analizar el efecto de las diferentes perturbaciones y de entradas, donde cobra especial relevancia la administración de insulina exógena para retornar a unas condiciones óptimas. La interacción entre estas variables se recoge como subobjetivo.
    \item Estudio de efecto de los sistemas de control glucémico en el organismo mediante el diseño de reguladores sencillos. Estas estrategias son la base para la implementación futura de sistemas regulatorios más complejos, como es el páncreas artificial. En este proyecto se emplea un regulador PID.
\end{enumerate}

\section{Objetivos personales}

\begin{enumerate}
    \item Profundizar en la comprensión del comportamiento de la glucosa e insulina, así como estudiar las variables que pueden influir en él.
    \item Iniciar el proceso de aprendizaje en el campo de la ingeniería de control, aplicando conocimientos matemáticos para establecer relaciones entre variables, y representarlas mediante simulaciones.
    \item Adquisición y dominio de LaTex como sistema de composición de texto para la creación de documentos técnicos.
    \item Comenzar este estudio que pueda ser empleado por otras personas en un futuro para el desarrollo de sistemas de control más sofisticados, o bien de modelos matemáticos innovadores.
\end{enumerate}


